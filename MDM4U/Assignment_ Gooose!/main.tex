    \documentclass{article}
\usepackage[utf8]{inputenc}
\usepackage{pgfplots}
\usepackage[utf8]{inputenc}
\usepackage[shortlabels]{enumitem}
\usepackage{amsmath} 
\usepackage{gensymb}
\usepackage{geometry}
\usepackage{listings}
\usepackage{xcolor}
\usepackage{pgfplots}
\usepackage{pgfplotstable}
\usepackage[linguistics]{forest}
\pgfplotsset{compat=1.7}
\usepackage{tikz}
\geometry{
 a4paper,
 total={170mm,257mm},
 left=20mm,
 top=20mm,
}
\title{Practice Assignment}
\author{Andy Yan}
\date{October 2022}

\begin{document}

\maketitle

\section{Question}
How many ways can you scramble all the letters in \textbf{GOOOSE}, such that \textbf{G} is always together with \textbf{O}, and \textbf{S} is never together with \textbf{E}? 

\section{Requirements}
\begin{enumerate}
\item Sketch a tree of counting plan. [3K]
\item Check your tree diagram to see if you have counted some possibilities more than once.
\item Provide an algebraic solution based on your counting plan. That is, multiply all numbers on a single branch to get the branch number, and then add up all the branch numbers. [4A]
\item As verification, list all possible scrambles in an organized way based on your counting plan. [3K] 
\end{enumerate}

\section{Counting Plan Tree}
We start with our most repeated letter \textbf{O}, then we insert \textbf{G} so we don't break the rules, and then we insert S so when we are counting we know where \textbf{E} can go. All scrambles are reversible starting from the second layer.
\begin{center}
\begin{forest}
    [OOO
        [GOOO
            [SGOOO]
            [GOSOO]
            [GOOSO]
            [GOOOS]
        ]
        [OGOO
            [SOGOO]
            [OSGOO]
            [OGSOO]
            [OGOSO]
            [OGOOS]
        ]
    ]
\end{forest}
\end{center}

\section{Solution 1}
From our counting plan, we are given all the scrambles (reversible) that don't contain \textbf{E}. We notice that all the scrambles that contain \textbf{G} in the first slot allow 3 possible slots for \textbf{E}. We also can see that if \textbf{G} and \textbf{S} are together there are only 3 possible slots for \textbf{E}. And for the rest of the cases where \textbf{G} and \textbf{S} are separated, there are only 4 possible slots for \textbf{E}. We also know that all of these scrambles are reversible.
\[\# of Scrambles = 2 \cdot (4 \cdot 3 + 2 \cdot 3 + 3 \cdot 4) = 2 \cdot 30 = 60\textbf{ Scrambles}\]
\section{Solution 2}
Without using our counting plan, we can solve using combinatorics. First we can group \textbf{G} and \textbf{O} together into one letter \textbf{GO} leaving us with 5 letters to work with. \\Our total scrambles is \(2 \cdot 5!\) since we can swap \textbf{G} and \textbf{O}. However, we must take into count that we can't have \textbf{S} and \textbf{E} next to each other. The number of ways \textbf{S} and \textbf{E} can be next to each other is \(2 \cdot \frac{5!}{2!}\). Now finally we have to divide everything by 2! because we have 3 of the same letter \textbf{O} and only two are identical since one is grouped with \textbf{O}. So our final formula is:
\[\frac{2 \cdot 5! - 2 \cdot \frac{5!}{2!}}{2!} = 60 \textbf{ Scrambles}\]


\section{List of Solutions}
\begin{center}
\def\arraystretch{1}
{\setlength{\tabcolsep}{2em}
\begin{tabular}{| c |} 
 \hline
egooos\\ 
egooso\\ 
egosoo\\ 
eogoos\\ 
eogoso\\ 
eogsoo\\ 
eoogos\\ 
eoogso\\ 
eooogs\\ 
eoosgo\\ 
eoosog\\ 
eosgoo\\ 
eosogo\\ 
eosoog\\ 
goeoos\\ 
goeoso\\ 
gooeos\\ 
goosoe\\ 
gosoeo\\ 
gosooe\\ 
oegoos\\ 
oegoso\\ 
oeogos\\ 
oeogso\\ 
oeoogs\\ 
oeosgo\\ 
oeosog\\ 
ogeoos\\ 
ogeoso\\ 
ogoeos\\ 
ogosoe\\ 
ogsoeo\\ 
ogsooe\\ 
ooegos\\ 
ooeogs\\ 
oogeos\\ 
oogsoe\\ 
oosgoe\\ 
oosoge\\ 
osgoeo\\ 
osgooe\\ 
osoego\\ 
osoeog\\ 
osogeo\\ 
osogoe\\ 
osooge\\ 
sgoeoo\\ 
sgooeo\\ 
sgoooe\\ 
soegoo\\ 
soeogo\\ 
soeoog\\ 
sogeoo\\ 
sogoeo\\ 
sogooe\\ 
sooego\\ 
sooeog\\ 
soogeo\\ 
soogoe\\ 
soooge\\
  \hline
\end{tabular}}
\end{center}
\end{document}
