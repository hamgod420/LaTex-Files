%======================================================================
\chapter{Introduction}
%======================================================================

Is there a correlation between one's height and heart rate? And if so would one's heart rate increase with height or the opposite? 

%----------------------------------------------------------------------
\section{Data Collection}
%----------------------------------------------------------------------

To research this topic, we collected data from primary sources by asking 50 17-year-old boys and girls from our school. We separated the data into boys and girls because girls tend to have varied heart rates from boys. We collected data in a way so that the distribution of height is well spread allowing us to concentrate on our topic. 

\section{Hypothesis}

Our hypothesis for this paper is: \textbf{the taller one is, the higher their heart rate should be.} We believe this because the larger one's body is, the more blood is needed to be pumped into the body which requires a higher heart rate.

\begin{table}[h]
    \section{General Data Analysis}
    \centering
    \caption{Main Data Part 1}
    \label{tab:table1}
\begin{tabular}{ cc }   % top level tables, with 2 columns
Males(Age 17)&\hspace{5em} Females(Age 17)\\  
% leftmost table of the top level table
\begin{tabular}{ ||c|c|| } 
\hline
Height(cm) & Heartbeat(bpm)\\ [0.5ex] 
\hline
\hline
163 & 75 \\
163 & 60 \\
164 & 75 \\
165 & 70 \\
165 & 72 \\
165 & 80 \\
166 & 75 \\
166 & 87 \\
167 & 79 \\
168 & 80 \\
168 & 76 \\
170 & 90 \\
170 & 70 \\
170 & 80 \\
171 & 69 \\
172 & 79 \\
172 & 92 \\
173 & 80 \\
173 & 76 \\
175 & 78 \\
175 & 87 \\
176 & 80 \\
176 & 85 \\
177 & 81 \\
178 & 81 \\
178 & 74 \\
\hline
\end{tabular} & \hspace{5em} % starting rightmost sub table
% table 2
\begin{tabular}{ ||c|c|| } 
\hline
Height(cm) & Heartbeat(bpm)\\ [0.5ex] 
\hline
\hline
160 & 69 \\
160 & 80 \\
160 & 80 \\
161 & 82 \\
162 & 90 \\
162 & 76 \\
162 & 84 \\
163 & 90 \\
163 & 85 \\
163 & 82 \\
163 & 100 \\
163 & 80 \\
163 & 96 \\
164 & 80 \\
164 & 86 \\
164 & 81 \\
165 & 72 \\
165 & 98 \\
165 & 86 \\
166 & 92 \\
166 & 82 \\
167 & 100 \\
168 & 89 \\
168 & 96 \\
168 & 85 \\
169 & 80 \\
\hline
\end{tabular} \\
\end{tabular}
\end{table}
\cleardoublepage

\begin{table}[h]
    \centering
    \caption{Main Data Part 2}
    \label{tab:table2}
\begin{tabular}{ cc }   % top level tables, with 2 columns
Males(Age 17) Continued&\hspace{5em} Females(Age 17) Continued\\  
% leftmost table of the top level table
\begin{tabular}{ ||c|c|| } 
\hline
Height(cm) & Heartbeat(bpm)\\ [0.5ex] 
\hline
\hline
180 & 68 \\
180 & 82 \\
180 & 84 \\
180 & 92 \\
181 & 91 \\
181 & 72 \\
181 & 76 \\
182 & 83 \\
182 & 76 \\
183 & 82 \\
183 & 83 \\
184 & 83 \\
185 & 84 \\
185 & 90 \\
186 & 88 \\
186 & 84 \\
188 & 90 \\
189 & 84 \\
190 & 86 \\
190 & 87 \\
193 & 92 \\
194 & 92 \\
195 & 95 \\
199 & 97 \\
\hline
\end{tabular} & \hspace{5em}  % starting rightmost sub table
% table 2
\begin{tabular}{ ||c|c|| } 
\hline
Height(cm) & Heartbeat(bpm)\\ [0.5ex] 
\hline
\hline
169 & 88 \\
170 & 90 \\
171 & 90 \\
172 & 92 \\
172 & 88 \\
172 & 92 \\
172 & 89 \\
173 & 91 \\
174 & 104 \\
174 & 89 \\
175 & 88 \\
175 & 96 \\
176 & 99 \\
176 & 91 \\
177 & 93 \\
178 & 94 \\
178 & 88 \\
178 & 97 \\
179 & 95 \\
180 & 97 \\
180 & 100 \\
181 & 100 \\
182 & 101 \\
183 & 93 \\
\hline
\end{tabular} \\
\end{tabular}
\end{table}
\\\\
To put the data into perspective, we will graph scatter plots with a line of best fit. 
\begin{minipage}[t]{0.5\textwidth}
\break 
\begin{center}
\break
    \textbf{Calculated Variables (Males)}
\end{center}
\[\mu = 81.44 \textbf{     (\ref{1})}\]
\[\sigma = 7.7648 \textbf{     (\ref{2})}\]
\[\sigma^2 = 60.2921 \textbf{     (\ref{3})}\]
\[r = 0.638 \textbf{     (\ref{5})}\]
\end{minipage}% % leave no gap
\begin{minipage}[t]{0.5\textwidth}
\break 
\begin{center}
\break
    \textbf{Calculated Variables (Females)}
\end{center}
\[\mu = 89.32 \textbf{     (\ref{9})}\]
\[\sigma = 7.8101 \textbf{     (\ref{10})}\]
\[\sigma^2 = 60.9977 \textbf{     (\ref{11})}\]
\[r = 0.6262 \textbf{     (\ref{12})}\]
\end{minipage}


\begin{figure}
    \caption{Scatter Plot Male}
    \label{fig:figure1l}
\end{figure}
\begin{tikzpicture}
\begin{axis}[
title = Height vs. Heartbeat in 17-year-old males,
xlabel={$Height(cm)$},
ylabel={$Heartbeat(bpm)$},
xmin = 160,xmax = 200,
ymax = 100,
]
\addplot+[
    only marks,
    scatter,
    mark size=2.9pt]
table[meta]
{data_male.txt};
]
\addplot[domain = 160:210, 
samples=100,
color=black,] {0.53171*x - 13.02362};

%ŷ = 0.53171X - 13.02362
\end{axis}
\end{tikzpicture}
\begin{center}
    \textbf{Line of Best Fit (Males):}
    \begin{equation}
    y = 0.5317x - 13.0236 \textbf{     (\ref{8})}
    \end{equation}
\end{center}
If we take a closer look at the graph, we can see that there is a moderate-to-strong positive linear correlation between the two variables being plotted. This is evident from the fact that the correlation coefficient (r) is equal to 0.638, which falls within the range of 0.50 to 0.70, which is indicative of a moderate-to-strong positive linear relationship. Upon further examination, we can also see that most of the data points tend to cluster around the line of best fit, which is a good indication that there is a strong relationship between the two variables being plotted. However, there are a few outliers that fall outside of the cluster, which could potentially indicate some variability or inconsistency in the data. Additionally, we can see that the highest heart rate in the data belongs to the tallest boy, and the lowest heart rate belongs to the shortest boy.


\begin{tikzpicture}
\begin{axis}[
title = Height vs. Heartbeat in 17-year-old females,
xlabel={$Height(cm)$},
ylabel={$Heartbeat(bpm)$},
xmin = 158,xmax = 184,
ymax = 105,
]
\addplot+[
    only marks,
    scatter,
    mark size=2.9pt]
table[meta]
{data_female.txt};
\addplot[domain = 158:184, 
samples=100,
color=black,] {0.72437*x - 33.54739};
]
\end{axis}
\end{tikzpicture}

\begin{figure}
    \caption{Scatter Plot Female}
    \label{fig:figure2}
\end{figure}
\begin{center}
    \textbf{Line of Best Fit (Females):}
    \begin{equation}
    y = 0.7244x - 33.5474 \textbf{     (\ref{16})}
    \end{equation}
\end{center}
Upon analyzing the graph for girls, we see that there is also a moderate-to-strong positive linear correlation between the two variables being plotted. This is evident from the fact that the correlation coefficient (r) is equal to 0.6262, which falls within the range of 0.50 to 0.70, indicating a moderate-to-strong positive linear relationship. From the graph, we can see that on the lower half most of the girls' heart-rates fall below the line of best fit, and the girls on the higher side fall on top.  
\\
\section{Grouped Data Analysis}
Grouped data analysis allows for more generalized height interval comparisons, helping us investigate the correlation.
\begin{center}
\begin{table}[h]
    \centering
    \caption{Grouped Heights}
    \label{tab:table3}
\end{table}
\vspace{0.5cm}
\\
\def\arraystretch{1}
{\setlength{\tabcolsep}{5em}
\begin{tabular}{|| c | c ||} 
\hline
\multicolumn{2}{||c||}{Grouped Frequency Distribution Table \textbf{Males}} \\
 \hline
 \hline
   Class Interval & Frequency\\ [0.5ex] 
 \hline
160 - 167 & 9\\
 \hline
168 - 175	& 12\\
 \hline
176 - 183	& 16\\
 \hline
184 - 191	& 9\\
 \hline
192 - 199	& 4\\
 \hline
  \textbf{Total:} & \textbf{50}\\
  \hline
\end{tabular}}
\end{center}

\begin{center}
\def\arraystretch{1}
{\setlength{\tabcolsep}{5em}
\begin{tabular}{|| c | c ||} 
\hline
\multicolumn{2}{||c||}{Grouped Frequency Distribution Table \textbf{Females}} \\
\hline
 \hline
   Class Interval & Frequency\\ [0.5ex] 
 \hline
160 - 164 & 16\\
 \hline
165 - 169	& 11\\
 \hline
170 - 174	& 9\\
 \hline
175 - 179	& 9\\
 \hline
180 - 184	& 5\\
 \hline
  \textbf{Total:} & \textbf{50}\\
  \hline
\end{tabular}}
\end{center}
\\\\
\\

\begin{table}[h]
    \centering
    \caption{Quartile Calculations}
    \label{tab:table4}
\end{table}

\begin{center}
    \begin{tabular}{ cc }   % top level tables, with 2 columns
    \def\arraystretch{1}
    \begin{tabular}{|| c | c | c | c | c | c ||}
    \hline
    \multicolumn{6}{||c||}{Male Quartile Calculations} \\
    \hline
    Interval & Q_1 & Q_2 & Q_3 & min & max\\
    \hline
    160-167 & 71 & 75 & 79.5 & 60 & 87\\ 
    168-175 & 76 & 79.5 & 83.5 & 69 & 92\\
    176-183 & 76 & 81.5 & 83.5 & 68 & 92\\
    184-191 & 84 & 86 & 89 & 83 & 90\\
    192-199 & 92 & 93.5 & 96 & 92 & 97\\
    \hline
    \end{tabular}
    
    \def\arraystretch{1}
    \begin{tabular}{|| c | c | c | c | c | c ||}
    \hline
    \multicolumn{6}{||c||}{Female Quartile Calculations} \\
    \hline
    Interval & Q_1 & Q_2 & Q_3 & min & max\\
    \hline
    160-164 & 80 & 82 & 88 & 69 & 100\\ 
    165-169 & 82 & 88 & 96 & 72 & 100\\
    170-174 & 89 & 90 & 92 & 88 & 104\\
    175-179 & 89.5 & 94 & 96.5 & 88 & 99\\
    180-184 & 95 & 100 & 100.5 & 93 & 101\\
    \hline
    \end{tabular}
\end{tabular}
\end{center}

\begin{figure}
    \centering
    \caption{Box Plots}
    \label{fig:figure2}
\end{figure}

\begin{center}
\begin{tikzpicture}
  \begin{axis}
    [
    title = \textbf{Box Plot Males},
    ytick={1,2,3,4,5},
    yticklabels={160-167, 168-175, 176-183, 184-191,192-199},
    ylabel={$Height(cm)$},
    xlabel={$Heartbeat(bpm)$},
    width=15cm,height=5.6cm,
    ]
    \addplot+[
    boxplot prepared={
      median=75,
      upper quartile=79.5,
      lower quartile=71,
      upper whisker=87,
      lower whisker=60
    },
    ] coordinates {};
    \addplot+[
    boxplot prepared={
      median=79.5,
      upper quartile=83.5,
      lower quartile=76,
      upper whisker=92,
      lower whisker=69
    },
    ] coordinates {};
    \addplot+[
    boxplot prepared={
      median=81.5,
      upper quartile=83.5,
      lower quartile=76,
      upper whisker=92,
      lower whisker=68
    },
    ] coordinates {};
    \addplot+[
    boxplot prepared={
      median=86,
      upper quartile=89,
      lower quartile=84,
      upper whisker=90,
      lower whisker=83
    },
    ] coordinates {};
    \addplot+[
    boxplot prepared={
      median=93.5,
      upper quartile=96,
      lower quartile=92,
      upper whisker=97,
      lower whisker=92
    },
    ] coordinates {};
  \end{axis}
\end{tikzpicture}

\begin{tikzpicture}
  \begin{axis}
    [
    title = \textbf{Box Plot Females},
    ytick={1,2,3,4,5},
    yticklabels={160-164, 165-169, 170-174, 175-179,180-184},
    ylabel={$Height(cm)$},
    xlabel={$Heartbeat(bpm)$},
    width=15cm,height=5.6cm,
    ]
    \addplot+[
    boxplot prepared={
      median=82,
      upper quartile=88,
      lower quartile=80,
      upper whisker=100,
      lower whisker=69
    },
    ] coordinates {};
    \addplot+[
    boxplot prepared={
      median=88,
      upper quartile=96,
      lower quartile=82,
      upper whisker=100,
      lower whisker=72
    },
    ] coordinates {};
    \addplot+[
    boxplot prepared={
      median=90,
      upper quartile=92,
      lower quartile=89,
      upper whisker=104,
      lower whisker=88
    },
    ] coordinates {};
    \addplot+[
    boxplot prepared={
      median=94,
      upper quartile=96.5,
      lower quartile=89.5,
      upper whisker=99,
      lower whisker=88
    },
    ] coordinates {};
    \addplot+[
    boxplot prepared={
      median=100,
      upper quartile=100.5,
      lower quartile=95,
      upper whisker=101,
      lower whisker=93
    },
    ] coordinates {};
  \end{axis}
\end{tikzpicture}

\end{center}
\\
\hspace{\parindent}A box plot is a useful tool for visualizing the distribution of a set of data and identifying any potential outliers. By analyzing our box plot, we can see that the majority of males have a heart rate between 70 beats per minute (bpm) and 85 bpm, while the majority of females have a heart rate between 85 bpm and 95 bpm. This is evident from the fact that the central boxes of the plots for each gender span these ranges, with the median heart rates falling near the center of the boxes. From both graphs, we can see that the majority of the data points are clustered within the mid-range of the heart rates, which suggests that there is a consistent relationship between the two variables being plotted.