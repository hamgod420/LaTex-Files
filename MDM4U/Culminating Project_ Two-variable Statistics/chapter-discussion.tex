%======================================================================
\chapter{Discussion}
%======================================================================

As seen throughout the introduction there seems to be a relationship that will be further discussed and analyzed in this section.

%----------------------------------------------------------------------
\section{Observations}
%----------------------------------------------------------------------
Firstly, the most evident observation to be made is: the average heart rate is increasing as the average height increases for this particular data set. Both correlations from males and females show a moderate-to-strong linear correlation due to both Pearson's R's  being very close to the boundary which shows a strong linear correlation (0.67). This statement is further supported by the box and whisker plots created.
Additionally, this proves the hypothesis to be true for this particular sample. In other words, for students in AY Jackson as height increases, heart rate also increases. 
\vspace{0.3cm}

Heart rate in this research experiment is labelled as the dependent variable because if it was independent, then it would not make sense for one to be taller just because they have a faster heart rate. Therefore, height is the independent variable and heart rate is the dependent variable. For this reason, this relationship is not a reverse cause-and-effect. The only possible relationships that can be drawn from this data are cause-and-effect and presumed relationships. The reason this relationship is not just cause-and-effect is because of the size of the data that was dealt with. Since the data size was not very large, this relationship cannot be fully justified as cause-and-effect especially since this is not the size of an entire population. There also could be hidden variables which will be addressed in the reflection tab (section 2.2).

%----------------------------------------------------------------------
\section{Reflection}
%----------------------------------------------------------------------
To conduct this experiment, a primary source of data collection was used to gather all the data. With this being said since there were only 2 people working on the collection, the data was obtained from a sample. This sample consisted of 50 seventeen-year-old males, and 50 seventeen-year-old females from the AY Jackson Secondary School. Since this sample does not reflect an entire population of 17-year-old males and females in all of Ontario, a definite conclusion cannot be drawn for a population, but it can be drawn for the sample from which the data originated.
\vspace{0.3cm}

Some very important and crucial pieces of information that were taken into consideration during this statistical experiment are the 3 main hidden variables. The first variable that was also kept constant was the age of all the people who were interviewed. A person's heart rate can change with age, so for this reason, the surveyed group needed to all have the same age. This made data collection slightly more challenging, but the results were more accurate. Another hidden variable would be the elevation of where students live. This is an uncontrolled variable, but since the sample was chosen from AY Jackson Secondary School, all of the students live in a general area where elevation does not play a major role. This made it negligible for the chosen sample but must be considered for national and international applications. Finally, the most important hidden variable would be activity and fitness levels. Students who have high fitness levels will have lower heart rates than others, which can account for the variations seen in the heart rate data. Likewise, health conditions also have a high chance of affecting this data, so that needs to be taken into account as well. 
\vspace{0.3cm}

Overall, the results of this investigation prove to be fairly accurate due to the calculated Pearson's R and the careful and well-thought experimenting process. There was not much Bias that partook during the surveying since there were no leading or loaded questions. There were some non-response biases due to the voluntary response method of surveying, but for ethical and personal reasons forced participation cannot be used. 
\vspace{0.3cm}

To further improve this analysis, a more universal surveying process of data can be used to gather more data which will better fit the size of a population. This will lead to a definite resolution to this relationship. Another method of improving this research project is to have a more refined and detailed filtering system where activity levels, elevation levels, and health conditions are used as parameters to organize and group the data into sections that reflect these various different populations.