\documentclass{article}
\usepackage[utf8]{inputenc}
\usepackage{pgfplots}
\usepackage[utf8]{inputenc}
\usepackage[shortlabels]{enumitem}
\usepackage{amsmath} 
\usepackage{gensymb}
\usepackage{geometry}
\usepackage{listings}
\usepackage{xcolor}
\usepackage{pgfplots}
\usepackage{pgfplotstable}
\pgfplotsset{compat=1.7}
\usepackage{tikz}
\geometry{
 a4paper,
 total={170mm,257mm},
 left=20mm,
 top=20mm,
}
\title{Practice Assignment: Combination with Identical
Elements}
\author{Andy Yan}
\date{October 2022}

\begin{document}

\maketitle

\section{Question}
How many different sums of money can be
formed from one \$2 bill, three \$5 bills, two
\$10 bills, and one \$20 bill?

\section{Approach}
To find the number of distinct ways we can pick from the bills is simple. We can group the bills accordingly to their value.
\[\{(\$2),(\$5, \$5, \$5),(\$10, \$10),(\$20)\}\]
To find the total number of distinct possible picks we, multiply the number of ways to pick from each type of bill. For example, there are 2 ways to pick the \$2 bill, either don't pick it or pick it. Another example is there are 4 ways to pick the \$5 bill, no pick, 1 bill, 2 bills, or 3 bills. So we find the number of ways to pick the other bills and multiply them.
\[2\cdot4\cdot3\cdot2 = 48 \textbf{ distinct ways}\]
To summarize this formula, we can let \(n_i\) represent the total numbers in a subset and add 1 to include the option of no selection.
\[(n_1 + 1)(n_2 + 1)...(n_i + 1)...\]

\section{Solution}

However, 48 distinct ways to pick from the bills don't give us 48\textbf{ distinct sums}. For example, picking \(\{\$10,\$10\}\) gives us the same sum as picking \(\{\$20\}\). To solve this problem, we can combine the excess \$5 bills and \$10 bill into a \$20 bill \(\$5 + \$5 + \$10 = \$20\). By doing this, we eliminate the possibility of the smaller bills creating duplicate values since the sum of 2,5,10 is smaller than 20. Our new set looks like this:
\[\{(\$2),(\$5),(\$10),(\$20, \$20)\}\]
Now we can apply the formula, however, we have to subtract 1 to avoid the null set case (\$0 sum):
\[(1 + 1)(1 + 1)(1 + 1)(2 + 1) - 1 = 2\cdot2\cdot2\cdot3 - 1 = 24 - 1 = 23 \textbf{ distinct sums}\]
Therefore, there are \textbf{23 different sums of money} formed from one \$2 bill, three \$5 bills, two
\$10 bills, and one \$20 bill?

\end{document}